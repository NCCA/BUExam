% Note this requires the (standard) exam package
\documentclass[a4paper,12pt,addpoints]{exam} 

%This is the package which you need to include
\usepackage{buexam}

% The Unit title of the exam (append RESIT to the end if necessary)
\setunittitle{Zombies in Popular Media and Mathematics}
% A shortened / abbreviated name of the unit
\setunittitleshort{ZPMM}
% The unit reference (might be the same as short name)
\setunitrefs{ZPMM}
% The level of the examination (4,5,6 etc)
\setunitlevel{4}
% The exam reference (get this from the school
\setexamref{ZZZ1234Z}
% The date of the exam
\setexamdate{10/06/2013}
% Unit leader of the course
\setunitleadername{Richard Southern}
% Unit leader extension
\setunitleaderext{61877}
% Assistant name
\setassistantname{Ben Ellis}
% Assitant extension
\setassistantext{65745}

% There are the specific instructions for candidates
% \renewcommand{\instructionstocandidates}{
% \begin{itemize}
% \item There are \textbf{THREE} questions
% \item Answer \textbf{ALL} questions
% \item All questions carry equal marks
% \item \textbf{NO} calculators, laptops or any other electronic equipment are permitted
% \item \textbf{NO} additional materials may be used
% \end{itemize}
% }

% This option should be commented out for the final paper (don't forget!)
\printanswers

\begin{document}

% Note that this command will create ``Section B\\Answer All Questions'' - you have to use this, as it is in an itemized environment.
\sectioninstructions{Answer ALL Questions}

% Questions can consist of multiple parts. Note that the sum of the parts must make the total (in this case, [15]).
\titledquestion{Zombies in films}
Consider Zombies in films. 
\begin{parts}
\part[5] Name some films with Zombies \droppoints
\begin{solution}
\begin{itemize}
\item Pride and Prejudice... with Zombies!
\item Abraham Lincoln... Zombie Hunter!
\item etc.
\end{itemize}
\end{solution}

% Maths example
\part[10] State the average number of characters eaten by a single Zombie in "`28 Days Later"'. Use the following formula:
\begin{align}
\bar{x} &= \frac{1}{n}\cdot \sum_{i=1}^{n} x_i \nonumber
\end{align}\droppoints
\begin{solution}
At least 3. Zombies are awesome.
\end{solution}
\end{parts}

% A newpage is required between sections
\newpage

% Note that this command will create ``Section B\\Answer Only ODD numbered questions'' - you have to use this, as it is in an itemized environment.
\sectioninstructions{Answer Only ODD numbered questions}

\titledquestion{Zombie Programming}
Consider the following code segment:
\begin{CPP}
#include <iostream>
using namespace std;

int main ()
{
  cout << "Hello World!";
  return 0;
}
\end{CPP}
In your own words, explain:
\begin{parts}
%Example of code inclusion
\part[5] Was this code produced by a Zombie? If so, why?\droppoints
\begin{solution}
Clearly, the lameness indicates that it is Zombie generated code (ZGC).
\end{solution}

\part[5] How would you improve on the above code? \droppoints
\begin{solution}
Don't use the std namespace! Better to give your fingers exercise and type it every time.
\end{solution}
\end{parts}
\newpage

\titledquestion{Zombie Algorithms}
\begin{parts}

% Example of an algorithm
 \part[10] Consider the following algorithm:

\begin{algorithmic}
\REQUIRE $n \geq 0$
\ENSURE $y = x^n$
\STATE $y \leftarrow 1$
\STATE $X \leftarrow x$
\STATE $N \leftarrow n$
\WHILE{$N \neq 0$}
\IF{$N$ is even}
\STATE $X \leftarrow X \times X$
\STATE $N \leftarrow N / 2$
\ELSE[$N$ is odd]
\STATE $y \leftarrow y \times X$
\STATE $N \leftarrow N - 1$
\ENDIF
\ENDWHILE
\end{algorithmic}
Describe in your own words the thought processes required for a Zombie to generate this algorithm. How can the promise of fresh brains serve as a motivators? \droppoints
\begin{solution}
Blah blah blah.
\end{solution}

% Example of image inclusion
\part[5] Admire our institution's mighty logo:
\begin{center}
\includegraphics[width=0.1\textwidth]{bulogo.pdf}
\end{center}
Draw this in a manner similar to a Zombie. Coloured pens are provided.\droppoints
\begin{solution}
Blah blah blah.
\end{solution}
\end{parts}

% The postamble is needed as part of the uni style. (this cannot be automatic as AtEndOfDocument adds a \newpage)
\afterlastquestion

\end{document}


